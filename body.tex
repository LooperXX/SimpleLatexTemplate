\part{正文} % 部分标题
% \chapter{章标题}这一章我们介绍这些内容。
\section{章节编排} % 节标题
\subsection{小节标题}这一小节我们介绍这些内容。
\subsubsection{子节标题}这一子节我们介绍这些内容。
\paragraph{段标题}这一段我们介绍这些内容。
\subparagraph{小段标题}这一小段我们介绍这些内容。

此外,在如上命令后添加*,则该部分部分自动编号,且不会编入自动目录中

\section{行文}

\subsection{作者}
\begin{lstlisting}[language={TeX}]
    \usepackage{authblk}
    \title{More than one Author with different Affiliations}
    \author[a]{Author A}
    \author[a]{Author B}
    \author[a]{Author C \thanks{Corresponding author: email@mail.com}}
    \author[b]{Author D}
    \author[b]{Author E}
    \affil[a]{Department of Computer Science, \LaTeX\ University}
    \affil[b]{Department of Mechanical Engineering, \LaTeX\ University}
    % 使用 \thanks 定义通讯作者
    \renewcommand*{\Affilfont}{\small\it} % 修改机构名称的字体与大小
    \renewcommand\Authands{ and } % 去掉 and 前的逗号
    \date{} % 去掉日期
\end{lstlisting}


\subsection{字体}

%字体族设置(罗马字体、无衬线字体、打印机字体)
\textrm{Roman Family} \textsf{Sans Serif Family} \texttt{Typewriter Family}    

{\rmfamily Roman Family} {\sffamily Sans Serif Family} {\ttfamily Typewriter Family}

{\sffamily who you are? you find self on everyone around. take you as the same as others!}

{\ttfamily Are you wiser than others? definitely on. in some days, my it is true. What can you achieve? a luxurious house? a brillilant car? an admirable career? who knows?}

%字体系列设置(粗细、宽度)
\textmd{Medium Series} \textbf{Boldface Series}

{\mdseries Medium Series} {\bfseries Boldface Series}

%字体形状(直立、斜体、伪斜体、小型大写)
\textup{Upright Shape} \textit{Italic Shape} \textsl{Slanted Shape} \textsc{Small Caps Shape}

{\upshape Upright Shape} {\itshape Italic Shape} {\slshape Slanted Shape} {\scshape Small Caps Shape}

%中文字体
{\songti 宋体} \quad {\heiti 黑体} \quad {\fangsong 仿宋} \quad {\kaishu 楷书}

中文字体的 \textbf{粗体} 和 \textit{斜体} 。


%字体大小,根据normalsize的大小确定,normalsize 在文档类的参数决定
{\tiny           Hello}\\
{\scriptsize     Hello}\\
{\footnotesize   Hello}\\
{\small          Hello}\\
{\normalsize      Hello}\\
{\large          Hello}\\
{\Large          Hello}\\
{\LARGE          Hello}\\
{\huge           Hello}\\
{\Huge           Hello}\\

%中文字号设置命令、
\zihao{-0} 你好!
\zihao{5} 你好!

\subsection{目录}
\begin{lstlisting}[language={tex}]
    \phantomsection % 生成新增目录项的链接
    \addcontentsline{toc}{section}{摘要} % 以section的名义向目录中添加条目
\end{lstlisting}

\subsubsection{编号设置}
修改编号的形式
\begin{lstlisting}[language={TeX}]
    \renewcommand\thesection{\arabic{section}} 
    % arabic 阿拉伯数字 
    % roman 小写的罗马数字 
    % Roman 大写的罗马数字 
    % alph 小写字母 
    % Alph 大写字母
    
    同样的,也可以修改子目录的编号方式,而且各目录编号方式可以不同
    \renewcommand\thesection{\Alph{section}} 
    \renewcommand\thesection{\arabic{subsection}}
\end{lstlisting}
其实,Latex内部有23个计数器,17个为序号计数器,6个是控制计数器,选列如下,因此我们可以根据需要改变计数器计数形式。
\begin{table}[h]
    \centering
    \begin{tabular}{|c|c|}
    \hline
    计数器名          & 用途         \\ \hline
    part          & 部序号        \\ \hline
    chapter       & 章序号        \\ \hline
    section       & 节          \\ \hline
    subsection    & 小节         \\ \hline
    subsubsection & 小小节        \\ \hline
    paragraph     & 段          \\ \hline
    subparagraph  & 小段         \\ \hline
    figure        & 插图序号       \\ \hline
    table         & 表格序号       \\ \hline
    equation      & 公式序号       \\ \hline
    page          & 页码计数器      \\ \hline
    footnote      & 脚注序号       \\ \hline
    mpfootnote    & 小页环境中脚注计数器 \\ \hline
    \end{tabular}
    \caption{计数器}
    \label{tab:mytab0}
\end{table}

\subsection{列表}

\paragraph{无序列表}
\begin{itemize}
  \item Javascript
  \item Html
  \item CSS
\end{itemize}

\paragraph{有序列表}
\begin{enumerate}
  \item Javascript
  \item Html
  \item CSS
\end{enumerate}

\paragraph{更加紧凑的无序列表}
\begin{compactitem}
    \item Javascript
    \item Html
    \item CSS
\end{compactitem}

\paragraph{更加紧凑的有序列表}
\begin{compactenum}
    \item Javascript
    \item Html
    \item CSS
\end{compactenum}

\paragraph{描述列表}
\begin{description}
  \item[Javascript] Javascript
  \item[Html] Html
  \item[CSS] CSS
\end{description}


\subsection{换行、分段与分页}
\begin{itemize}
    \item $\backslash \backslash$ 表示强制换行 与 $\backslash newline$ 类似但有不同
    \item $\backslash \backslash[offset]$ 表示强制换行并且与下一行的行间距为原来行间距+offset
    \item $\backslash linebreak$ 强制换行,与$\backslash newline$的区别为其当前行分散对齐。
    \item $\backslash par$ 表示强制分段
    \item $\backslash newpage$ 分页命令。
    \item $\backslash clearpage$ 分页命令
\end{itemize}

\begin{lstlisting}
    \newpage:  The \newpage command ends the current page.
    \clearpage:The \clearpage command ends the current page and causes all figures and tables that have so far appeared in the input to be printed.
\end{lstlisting}

通俗点讲就是当你新加的一页内容较多时,两者基本一样,当新加的一页内容较少时,$\backslash newpage$就无法实现你想要的效果,但$\backslash clearpage$可以。


\subsection{空格}
$\backslash quad, 1em, em, m$ 代表当前字体下接近字符‘M’的宽度。

\begin{table}[h]
    \centering
    \begin{tabular}{|c|c|}
        \hline
        两个quad空格 & a \qquad b \\
        \hline
        一个quad空格 & a \quad b \\
        \hline
        大空格	 & a\ b \\
        \hline
        中等空格 &	a\;b \\
        \hline
        小空格 & a\,b \\
        \hline
        没有空格 & ab \\
        \hline
        紧贴    &  a\!b  \\
        \hline
    \end{tabular}
    \caption{空格效果}
    \label{tab:mytab1}
\end{table}

\subsection{缩进}
$\backslash noindent$ 可以强制首行不缩进

$\backslash setlength \{\backslash parindent\}\{1em\}$ 自行设置缩进量

\subsection{引用}
在bib文件中写入数据,并以$\backslash cite{}$引用,在文档最后呈现所有参考文献。

引用共三种 $cite,url/href,ref$ 分别是
\begin{itemize}
    \item 对bib中的引用 \cite{W:liam} \cite{W:jingwhale}
    \item 对网址的直接引用 \url{www.baidu.com} / \href{www.baidu.com}{百度}
    \item 对label的引用 如图\ref{fig:myfig1}
\end{itemize}

此外,可以使用$\backslash footnote \{脚注内容\}$添加脚注(标题中添加脚注时需要在在前面多加$\backslash protect$ ),脚注标号设置如下
\begin{lstlisting}[language={TeX}]
    % 符号/字母 脚注标号设置
    \renewcommand{\thefootnote}{\fnsymbol{footnote}} % 重置\footnote{}命令为符号脚注
    \renewcommand{\thefootnote}{\alph{footnote}} % 重置\footnote{}命令为符号脚注
    
    % arabic 阿拉伯数字 
    % roman 小写的罗马数字 
    % Roman 大写的罗马数字 
    % alph 小写字母 
    % Alph 大写字母
    
    % 数字脚注标号设置,且标号可以略去
    \usepackage{lipsum}
    \newcommand\blfootnote[1]{% 
    \begingroup 
    \renewcommand\thefootnote{}\footnote{#1}% 
    \addtocounter{footnote}{-1}% 
    \endgroup 
    }
    
    \blfootnote{*Corresponding Author} % 无标号
    \footnote{Corresponding Author} % 正常显示标号,从1开始
\end{lstlisting}


\subsection{代码}
使用listings与xcolor实现代码段的插入,以下是配置示例
\begin{lstlisting}[language={TeX}]
\usepackage{listings} % 代码段
\usepackage{xcolor} % 颜色

\lstset{ %
  backgroundcolor=\color{white},   % choose the background color; you must add \usepackage{color} or \usepackage{xcolor}
  basicstyle=\ttfamily,            % the size of the fonts that are used for the code
  breakatwhitespace=false,         % sets if automatic breaks should only happen at whitespace
  breaklines=true,                 % sets automatic line breaking
  captionpos=b,                    % sets the caption-position to bottom
  commentstyle=\ttfamily\color{mygreen},    
                                   % comment style
  deletekeywords={},               % if you want to delete keywords from the given language
  escapeinside={},                 % if you want to add LaTeX within your code
  extendedchars=true,              % lets you use non-ASCII characters; for 8-bits encodings only, does not work with UTF-8
  frame=single,                    % adds a frame around the code
  keepspaces=true,                 % keeps spaces in text, useful for keeping indentation of code (possibly needs columns=flexible)
  keywordstyle=\color{blue},       % keyword style
  language=C++,                    % the language of the code
  morekeywords={},                 % if you want to add more keywords to the set
  numbers=left,                    % where to put the line-numbers; possible values are (none, left, right)
  numbersep=5pt,                   % how far the line-numbers are from the code
  numberstyle=\tiny\color{mygray}, % the style that is used for the line-numbers
  rulecolor=\color{black},         % if not set, the frame-color may be changed on line-breaks within not-black text (e.g. comments (green here))
  showspaces=false,                % show spaces everywhere adding particular underscores; it overrides 'showstringspaces'
  showstringspaces=false,          % underline spaces within strings only
  showtabs=false,                  % show tabs within strings adding particular underscores
  stepnumber=1,                    % the step between two line-numbers. If it's 1, each line will be numbered
  stringstyle=\color{mymauve},     % string literal style
  tabsize=2,                       % sets default tabsize to 2 spaces
  title=\lstname                   % show the filename of files included with \lstinputlisting; also try caption instead of title
}
\end{lstlisting}

\subsection{算法}

\section{数学公式}

LaTeX 的数学模式有两种:行内模式 (inline) 和行间模式 (display)。前者在正文的行文中,插入数学公式;后者独立排列单独成行,并自动居中。

在行文中,使用 $ X^2 $ 可以插入行内公式,使用 \[ X^2 \] 可以插入行间公式,如果需要对行间公式进行编号,则可以使用 equation 环境(equation表示不需要编号):

\begin{equation}
\sqrt{x}, \frac{1}{2}
\end{equation}

\subsection{运算符} 

一些小的运算符,可以在数学模式下直接输入;另一些需要用控制序列生成,如
\[ \pm\; \times \; \div\; \cdot\; \cap\; \cup\;
\geq\; \leq\; \neq\; \approx \; \equiv \]

连加、连乘、极限、积分等大型运算符 % \sum, \prod, \lim, \int
他们的上下标在行内公式中被压缩,以适应行高。我们可以强制显式地指定是否压缩这些上下标。例如:

$ \sum_{i=1}^n i\quad \prod_{i=1}^n $ 

$ \sum\limits _{i=1}^n i\quad \prod\limits _{i=1}^n $

\[ \lim_{x\to0}x^2 \quad \int_a^b x^2 dx \]
\[ \lim\nolimits _{x\to0}x^2\quad \int\nolimits_a^b x^2 dx \]

多重积分
\[ \iint\quad \iiint\quad \iiiint\quad \idotsint \]

\subsection{定界符}

\[ \Biggl(\biggl(\Bigl(\bigl((x)\bigr)\Bigr)\biggr)\Biggr) \]
\[ \Biggl[\biggl[\Bigl[\bigl[[x]\bigr]\Bigr]\biggr]\Biggr] \]
\[ \Biggl \{\biggl \{\Bigl \{\bigl \{\{x\}\bigr \}\Bigr \}\biggr \}\Biggr\} \]
\[ \Biggl\langle\biggl\langle\Bigl\langle\bigl\langle\langle x
\rangle\bigr\rangle\Bigr\rangle\biggr\rangle\Biggr\rangle \]
\[ \Biggl\lvert\biggl\lvert\Bigl\lvert\bigl\lvert\lvert x
\rvert\bigr\rvert\Bigr\rvert\biggr\rvert\Biggr\rvert \]
\[ \Biggl\lVert\biggl\lVert\Bigl\lVert\bigl\lVert\lVert x
\rVert\bigr\rVert\Bigr\rVert\biggr\rVert\Biggr\rVert \]

\subsection{省略号}
\[ x_1,x_2,\dots ,x_n\quad 1,2,\cdots ,n\quad
\vdots\quad \ddots \]

\subsection{矩阵}
\[ \begin{pmatrix} a&b\\c&d \end{pmatrix} \quad
\begin{bmatrix} a&b\\c&d \end{bmatrix} \quad
\begin{Bmatrix} a&b\\c&d \end{Bmatrix} \quad
\begin{vmatrix} a&b\\c&d \end{vmatrix} \quad
\begin{Vmatrix} a&b\\c&d \end{Vmatrix} \]

行内的小矩阵:Marry has a little matrix $ ( \begin{smallmatrix} a&b\\c&d \end{smallmatrix} ) $.

\subsection{多行公式}

\subsubsection{长公式}
不对齐
\begin{multline}
x = a+b+c+{} \\
d+e+f+g
\end{multline}
对齐
\[\begin{aligned}
x ={}& a+b+c+{} \\
&d+e+f+g
\end{aligned}\]

\subsubsection{公式组}
\begin{gather}
a = b+c+d \\
x = y+z
\end{gather}

\begin{align}
a &= b+c+d \\
x &= y+z
\end{align}

\subsubsection{分段公式}
\begin{equation}
    y= \begin{cases}
    -x,\quad x\leq 0 \\
    x,\quad x>0
    \end{cases} 
\end{equation}

\subsubsection{复杂公式}
\begin{equation}
\left.
\begin{aligned}
x+y &> 5 \\
y-y &> 11
\end{aligned}
\ \right\}\Rightarrow x^2 - y^2 > 55
\end{equation}

\begin{equation}
  F = 
  \begin{cases}
    n + b + 1 & \text{if $a \neq \emptyset$}\\

    \tilde{z} = \operatornamewithlimits{argmin}\limits_{\tilde{z}}\operatorname{dist}(z, \tilde{z}) & \text{otherwise}
  \end{cases}
\end{equation}

\begin{equation}
    \ell_{\mathrm{uniform}}\triangleq\log \underset{~~~x, y\stackrel{i.i.d.}{\sim} p_{\mathrm{data}}}{\mathbb{E}}   e^{-2\Vert f(x)-f(y) \Vert^2}
\end{equation}

% \stackrel 1在2上面 \underset 1在2下面 两个命令的1都小
\begin{equation}
    \stackrel{A}{B} \\ 
    \underset{B}{A}
\end{equation}

% 也可以通过 operatornamewithlimits 
\begin{align}
    \operatorname{softmax}_A, \operatornamewithlimits{softmax}_A
\end{align}


\subsubsection{数学符号}

\checkmark, \cmark ,\xmark

\section{图表}
$\backslash centerline{}$ 以及 $\backslash centering$可以居中图表

\subsection{插入图片}
如图\ref{fig:myfig1}
\begin{figure}[htbp]
\centering
\includegraphics[width = .8\textwidth]{figure/Latex.jpg}
\caption{图片标题}
\label{fig:myfig1}
\end{figure}

\subsection{插入表格}
如表\ref{tab:mytab2},tabular环境提供了最简单的表格功能。它用 $\backslash hline$ 命令表示横线,在列格式中用 | 表示竖线;用 $\&$ 来分列,用 $\backslash\backslash$换行 

Latex的表格编写实在是太繁琐了,推荐使用 \textbf{\href{http://www.tablesgenerator.com/}{Tables Generator}} 轻松愉悦的完成表格 :>

\begin{table}[htbp]
    \centering
    \begin{tabular}{|l|c|r|}
         \hline
        操作系统& 发行版& 编辑器\\
         \hline
        Windows & MikTeX & TexMakerX \\
         \hline
        Unix/Linux & teTeX & Kile \\
         \hline
        Mac OS & MacTeX & TeXShop \\
         \hline
        通用& TeX Live & TeXworks \\
         \hline
    \end{tabular}
    \caption{表格标题}
    \label{tab:mytab2}
\end{table}

\begin{table}[h!]
    \centering
      \begin{tabular}{ccc}
        \toprule 
        \textbf{Column 1} & \textbf{Column 2} & \textbf{Column 3}\\
        \midrule
        A & 10.23 & a\\
        B & 45.678 & b\\
        C & 99.987 & c\\
        \bottomrule
      \end{tabular}
    \caption{三线表}
    \label{tab:mytab2_}
\end{table}

% Thanks to https://github.com/TobiasLee/Offiziersmesser/blob/main/latex/table.md !
\begin{table}[t!]
    \centering
    \small 
    \begin{tabularx}{0.5\linewidth}{X|c} % use X to denote where need auto break and specify the width 
    \toprule
      Input   &  Label \\
    \midrule
     \textbf{Text A}:  Organs are collections of tissues grouped together performing a common function.
  &  1  \\ 
   \midrule  
  \textbf{Text A}: Organs are collections of tissues grouped together performing a common function. 
  \textbf{Text B}: Text B: Does this sentence contains a definition? 
& 1 \\ 
    \bottomrule
    \end{tabularx}
    \caption{Caption}
    \label{tab:my_label}
\end{table}

\subsubsection{跨列表格}
如表\ref{tab:mytab3}
\begin{table}[htbp]
    \centering
    \begin{tabular}{|l|c|r|}
      \hline
      % after \\: \hline or \cline{col1-col2} \cline{col3-col4} ...
      左列 & 中列 & 右列 \\
      \hline
      2行1列 & 2行2列 & 2行3列 \\
      \hline
      \multicolumn{2}{|c|}{跨越2015} & 3行3列 \\
      \hline
      4行1列 & 4行2列 & 4行3列 \\
      \hline
    \end{tabular}
    \caption{跨列表格}
    \label{tab:mytab3}
\end{table}

\subsection{浮动效果}
 htbp 选项用来指定插图的理想位置,这几个字母分别代表 here, top, bottom, float page,也就是就这里、页顶、页尾、浮动页(专门放浮动体的单独页面或分栏)。$\backslash centering$ 用来使插图居中;$\backslash caption$ 命令设置插图标题,LaTeX 会自动给浮动体的标题加上编号。注意 $\backslash label$ 应该放在标题命令之后。

\subsection{图表混合}
\subsubsection{一行多图}
在导言区引入以下包,使用subfigure即可实现一行多图等效果,并且图片的标题与label都可以相应设置。子图的标题中的字母可以通过设置修改
\begin{lstlisting}[language={TeX}]
    \usepackage{graphicx}
    \usepackage{caption}
    \usepackage{subcaption}
\end{lstlisting}
$\backslash quad$分开两张图片,$\backslash quad$之后再加上回车,则分行

\begin{figure}[htbp]
    \centering
    \begin{subfigure}[t]{0.2\linewidth}
        \includegraphics[width=\linewidth]{figure/Latex.jpg}
        \caption{一行多图}
        \label{fig:myfig2}
    \end{subfigure}
    \quad
    \begin{subfigure}[t]{0.2\linewidth}
        \includegraphics[width=\linewidth]{figure/Latex.jpg}
        \caption{一行多图}
        \label{fig:myfig3}
    \end{subfigure}
    \quad
    \begin{subfigure}[t]{0.2\linewidth}
        \includegraphics[width=\linewidth]{figure/Latex.jpg}
        \caption{一行多图}
        \label{fig:myfig4}
    \end{subfigure}
    \quad
    \begin{subfigure}[t]{0.2\linewidth}
        \includegraphics[width=\linewidth]{figure/Latex.jpg}
        \caption{一行多图}
        \label{fig:myfig5}
    \end{subfigure}
    \caption{一行多图}
    \label{fig:myfig6}
\end{figure}

\begin{figure}[htbp]
    \centering
    \begin{subfigure}[t]{0.4\linewidth}
        \includegraphics[width=\linewidth]{figure/Latex.jpg}
        \caption{多行多图}
        \label{fig:myfig7}
    \end{subfigure}
    \quad
    \begin{subfigure}[t]{0.4\linewidth}
        \includegraphics[width=\linewidth]{figure/Latex.jpg}
        \caption{多行多图}
        \label{fig:myfig8}
    \end{subfigure}
    \quad
    
    \begin{subfigure}[t]{0.4\linewidth}
        \includegraphics[width=\linewidth]{figure/Latex.jpg}
        \caption{多行多图}
        \label{fig:myfig9}
    \end{subfigure}
    \quad
    \begin{subfigure}[t]{0.4\linewidth}
        \includegraphics[width=\linewidth]{figure/Latex.jpg}
        \caption{多行多图}
        \label{fig:myfig10}
    \end{subfigure}
    \caption{多行多图}
    \label{fig:myfig11}
\end{figure}

\subsubsection{一行多表}
一行多表以及图表混合均使用minipage的方式实现,标题与label均独立,可以在\href{http://www.sascha-frank.com/latex-minipage.html}{这里}查看其文档

\begin{lstlisting}[language={TeX}]
    \begin{minipage}[adjusting]{width of the minipage}
     Text ... \ \
     Images ... \ \
     Tables ... \ \
    \end{minipage} 
\end{lstlisting}

\begin{table}[htbp]
  \begin{minipage}{0.45\linewidth}
    \centering
    \captionof{table}{一行多表}
    \begin{tabular}{|l|c|r|}
         \hline
        操作系统& 发行版& 编辑器\\
         \hline
        Windows & MikTeX & TexMakerX \\
         \hline
        Unix/Linux & teTeX & Kile \\
         \hline
        Mac OS & MacTeX & TeXShop \\
         \hline
        通用& TeX Live & TeXworks \\
         \hline
    \end{tabular}
    \label{tab:mytab4}
  \end{minipage}
  \quad
  \begin{minipage}{0.45\linewidth}
    \centering
    \captionof{table}{一行多表}
        \begin{tabular}{|l|c|r|}
             \hline
            操作系统& 发行版& 编辑器\\
             \hline
            Windows & MikTeX & TexMakerX \\
             \hline
            Unix/Linux & teTeX & Kile \\
             \hline
            Mac OS & MacTeX & TeXShop \\
             \hline
            通用& TeX Live & TeXworks \\
             \hline
        \end{tabular}
    \label{tab:mytab5}
  \end{minipage}
\end{table}

\subsubsection{图表混合}

\begin{figure}[htb]
    \centering
    \begin{minipage}[c]{0.4\linewidth}
        \centering
        \includegraphics[width=\linewidth]{figure/Latex.jpg}
        \caption{图表混合}
        \label{fig:myfig12}
    \end{minipage}
    \quad
    \begin{minipage}[c]{0.4\linewidth}
        \centering
        \begin{tabular}{|l|c|r|}
             \hline
            操作系统& 发行版& 编辑器\\
             \hline
            Windows & MikTeX & TexMakerX \\
             \hline
            Unix/Linux & teTeX & Kile \\
             \hline
            Mac OS & MacTeX & TeXShop \\
             \hline
            通用& TeX Live & TeXworks \\
             \hline
        \end{tabular}
        \captionof{table}{图表混合}
        \label{tab:mytab6}
    \end{minipage}
\end{figure}

\section{版面设置}

\subsection{段落对齐}

\begin{center}
对于生成复杂表格和数学公式,这一点表现得尤为突出。
\end{center}
\begin{flushleft}
LATEX(英语发音:/ˈleɪtɛk/ ),
\end{flushleft}
\begin{flushright}
是一种基于TEX的排版系统,
\end{flushright}

\subsection{页面边距}

设置页边距,推荐使用 geometry 宏包。可以在\textbf{\href{http://texdoc.net/texmf-dist/doc/latex/geometry/geometry.pdf}{这里}}查看它的说明文档。

例如将纸张的长度设置为 20cm、宽度设置为 15cm、左边距 1cm、右边距 2cm、上边距 3cm、下边距 4cm,后两行可以定制特定页的边距:
\begin{lstlisting}[language={TeX}]
    \usepackage{geometry}
    \geometry{papersize={20cm,15cm}}
    \geometry{left=1cm,right=2cm,top=3cm,bottom=4cm}
    \newgeometry{left = 0.8 cm, right = 0.8 cm, bottom = 0.8cm} % 单独设置从此行向下的页边距
    \restoregeometry % 还原页边距
\end{lstlisting}

\subsection{页眉页脚}
设置页眉页脚,推荐使用 fancyhdr 宏包。可以在\textbf{\href{http://texdoc.net/texmf-dist/doc/latex/fancyhdr/fancyhdr.pdf}{这里}}查看它的说明文档。

自定义页眉和页脚。
为页眉和页脚加上装饰性的横线。
页眉和页脚的宽度可以超过正文文本的宽度。
多行的页眉和页脚。
奇偶页使用不同格式的页眉和页脚。
每章的首页使用不同格式的页眉和页脚。
浮动对象页使用不同格式的页眉和页脚。
控制页眉和页脚的字体,包括字形,字族,大小写等。

比如我希望,在页眉左边写上我的名字,中间写上今天的日期,右边写上我的电话;页脚的正中写上页码;页眉和正文之间有一道宽为 0.4pt 的横线分割,可以在导言区加上如下几行:

\begin{lstlisting}[language={TeX}]
    \usepackage{fancyhdr}
    \pagestyle{fancy}
    \lhead{\author}
    \chead{\date}
    \rhead{}
    \lfoot{}
    \cfoot{\thepage}
    \rfoot{}
    \renewcommand{\headrulewidth}{0.4pt}
    \renewcommand{\headwidth}{\textwidth}
    \renewcommand{\footrulewidth}{0pt}
    
    \pagenumbering{数字形式} % arabic, 阿拉伯数字 roman, 小写罗马数字 Roman,大写罗马数字 alpha, 小写拉丁字母 Alpha, 大写拉丁字母
\end{lstlisting}

\subsection{行间距}
我们可以通过 setspace 宏包提供的命令来调整行间距。比如在导言区添加如下内容,可以将行距设置为字号的 1.5 倍:
\begin{lstlisting}[language={TeX}]
    \usepackage{setspace}
    \onehalfspacing
\end{lstlisting}

\subsection{段间距}
我们可以通过修改长度 $\backslash parskip$ 的值来调整段间距。例如在导言区添加以下内容

\begin{lstlisting}[language={TeX}]
    \addtolength{\parskip}{.4em}
\end{lstlisting}

则可以在原有的基础上,增加段间距0.4em。如果需要减小段间距,只需将该数值改为负值即可。

\bibliographystyle{plain} % 指定参考文献的呈现方式
\clearpage
\phantomsection
\addcontentsline{toc}{section}{参考文献} %向目录中添加条目,以章的名义
\bibliography{ref} % 命令用于指定之前生成的.bib库。